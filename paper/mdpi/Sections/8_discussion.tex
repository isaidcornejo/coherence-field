\section{Discussion}
\label{sec:discussion}

The scalar diagnostic introduced in this work provides a unified framework for
quantifying empirical deviations from Fisher--Rao geometry. Its spectral
representation in terms of the eigenvalues of $H = G^{-1}C$ allows for a
transparent geometric and statistical interpretation: empirical reinforcement
corresponds to $\lambda_i > 1$, suppression corresponds to $\lambda_i < 1$, and
the deviation from equilibrium aggregates into the single invariant
$\mathcal{A} = \sum_i(\lambda_i - 1)$.

\subsection{Interpretation of Empirical Alignment}

The interpretation of alignment through the Fisher-normalized spectrum reveals
how empirical data shape the effective sensitivity of a model:
\begin{itemize}
    \item Reinforced directions correspond to strong curvature induced by the
    data distribution, often reflecting dominant modes, coherent structures, or
    training-induced specialization in neural networks
    \cite{Sagun2016,Papyan2019,Laurent2018,Nakkiran2020}.
    \item Suppressed directions correspond to collapsed or constrained
    sensitivity, arising either from structural properties of the model (as in
    the Laplace family) or from data distributions that reduce empirical
    variability.
\end{itemize}

In this sense, the scalar diagnostic captures not only the magnitude but also
the \emph{balance} between these two opposing geometric tendencies.

\subsection{Comparison to Existing Notions of Dimensionality}

Traditional measures of effective dimensionality—participation ratio, spectral
entropy, effective rank—summarize properties of a covariance or Hessian matrix
but do not compare empirical sensitivity to the model’s intrinsic geometric
structure. The diagnostic $\mathcal{A}$ differs fundamentally in that it
quantifies \emph{empirical deformation relative to Fisher--Rao geometry}, making
it invariant under reparametrization and applicable across a broad range of
models.

Moreover, $\phi = \max\{\sqrt{\mathcal{A}},0\}$ isolates the excess-alignment
sector, which is often of particular interest in detecting dominant curvature
modes, optimization instabilities, or critical transitions during learning.

\subsection{Implications for Optimization and Learning Dynamics}

The presence of reinforced directions ($\lambda_i \gg 1$) suggests that empirical
curvature concentrates strongly along a small number of directions, a behavior
commonly associated with:
\begin{itemize}
    \item rapid convergence along principal gradient directions,
    \item stiff optimization dynamics,
    \item the emergence of low-dimensional effective subspaces,
    \item sensitivity to initialization or perturbations.
\end{itemize}

Conversely, suppressed directions ($\lambda_i \ll 1$) correspond to regions where
empirical curvature is weaker than geometric expectations, hinting at flat
directions, redundancies in parameterization, or underconstrained degrees of
freedom.

The scalar diagnostic therefore provides a compact summary of how learning or
sampling reshapes the geometric landscape of a model.

\subsection{Limitations and Scope}

Several limitations should be kept in mind:
\begin{itemize}
    \item Estimating $G$ and $C$ can be challenging in high dimensions, often
    requiring regularization, subsampling, or approximate score estimation.
    \item The scalar diagnostic summarizes only the trace structure of the
    alignment operator and does not capture the full eigenstructure of $H$.
    It should therefore be interpreted alongside (or informed by) spectral
    information when available.
    \item $\mathcal{A}=0$ does not imply $q=p$; rather, it indicates that the
    empirical and geometric traces agree. Only in minimal exponential families
    does vanishing $\mathcal{A}$ correspond to exact Fisher equilibrium
    \cite{Brown1986,WainwrightJordan2008}.
\end{itemize}

Despite these limitations, the diagnostic retains conceptual simplicity while
capturing essential geometric information spanning suppression, equilibrium, and
reinforcement regimes.

\paragraph{Trace cancellation.}
A vanishing diagnostic, $\mathcal{A}=0$, enforces only that the \emph{trace}
of the alignment operator equals that of the identity. Reinforced and suppressed
directions may cancel:
\[
\sum_{i=1}^{D} (\lambda_i - 1) = 0
\quad\not\Rightarrow\quad
\lambda_i = 1 \;\; \forall i.
\]
Thus $\mathcal{A}=0$ may occur even when $C \neq G$ and the empirical geometry
remains anisotropic. Minimal exponential families are the only class for which
$\mathcal{A}=0$ guarantees full Fisher equilibrium, since no cancellation across
directions is possible.

\paragraph{Scalability considerations.}
The computation of the alignment operator $H = G^{-1}C$ requires solving a
linear system with the Fisher metric and, in high-dimensional models, computing
(or approximating) its eigenvalues. While this is tractable for the low- and
moderate-dimensional settings considered here, large-scale neural networks may
require stochastic approximations, low-rank projections, or Lanczos-type
methods for estimating the dominant eigenvalues of $H$. Because $\mathcal{A}$
depends only on the trace of $H$, randomized trace estimators or Hutchinson-type
methods offer practical scalable alternatives in very high dimensions.

\subsection{Outlook}

The scalar diagnostic may have broader implications in:
\begin{itemize}
    \item monitoring the evolution of curvature during training,
    \item characterizing transitions between learning phases,
    \item informing natural gradient or curvature-aware optimization methods,
    \item studying generalization via curvature-based metrics,
    \item analyzing implicit models where score estimation is approximate.
\end{itemize}

The empirical experiments included in this work demonstrate that the diagnostic
remains stable, interpretable, and informative across classical models, mixture
models, and deep neural networks.

Future work may explore alignment flows, dynamical equations for alignment,
connections to information geometry in statistical physics, and the behavior of
$\mathcal{A}$ in large-scale, high-dimensional learning systems.
