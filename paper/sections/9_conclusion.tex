\section{Conclusion}
\label{sec:conclusion}

This work introduced a reparametrization-invariant scalar diagnostic for
quantifying empirical deviation from Fisher--Rao geometry. By expressing the
alignment operator $H = G^{-1}C$ in a Fisher-normalized form, we showed that its
spectrum admits real, non-negative eigenvalues whose deviation from unity
captures reinforced and suppressed empirical sensitivity. The resulting scalar
invariant,
\[
\mathcal{A} = \sum_{i=1}^D (\lambda_i - 1),
\qquad
\phi = \max\{\sqrt{\mathcal{A}},0\},
\]
provides a concise geometric summary of these effects.

Across a range of models—from exponential families
\cite{Brown1986,WainwrightJordan2008} to mixture models
\cite{McLachlanPeel2000} to trained neural networks—the diagnostic exhibited
consistent behavior aligned with theoretical expectations:
$\mathcal{A}\approx 0$ under equilibrium, $\mathcal{A}<0$ under suppression,
and $\mathcal{A}>0$ whenever empirical reinforcement was present. Deep networks,
in particular, displayed strong anisotropic curvature characterized by large
outlier eigenvalues and large positive values of $\mathcal{A}$
\cite{Sagun2016,Papyan2019,Laurent2018,Nakkiran2020}.

The scalar diagnostic sits alongside, but distinct from, existing measures of
effective dimensionality, offering a geometric perspective centered on deviation
from Fisher curvature rather than absolute spectral shape. Its invariance under
reparametrization and ease of computation make it suitable for monitoring
learning dynamics, analyzing curvature-induced phase transitions, or informing
curvature-aware optimization.

Future avenues include extending the diagnostic to implicit or score-based
models, studying dynamical flows of empirical alignment during training,
exploring high-dimensional asymptotics, and connecting alignment behavior to
generalization and robustness in modern learning systems.

All experiments and figures in this work are fully reproducible. 
Code is available at\\
\href{https://github.com/isaidcornejo/coherence-field}{github.com/isaidcornejo/coherence-field}.
