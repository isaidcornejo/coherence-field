\section{Introduction}

The Fisher--Rao metric endows parametric statistical models with a canonical
Riemannian structure that reflects infinitesimal statistical distinguishability
\cite{Fisher1922,Rao1945,AmariNagaoka2000,Amari2016}. Its geometric
interpretation underlies classical asymptotic theory, natural gradient
optimization \cite{Amari1998,Martens2014}, exponential-family structure
\cite{Brown1986,WainwrightJordan2008}, and the analysis of likelihood curvature
in information geometry.

In practice, empirical data seldom match the model distribution exactly.
Whenever the data-generating distribution $q(x)$ differs from the model
$p(x|\theta)$, empirical score statistics become anisotropic, leading to
heavy-tailed empirical Fisher matrices, dominant outlier eigenvalues, and
effective dimensionality collapse. These phenomena are now well documented in
high-dimensional models and deep networks
\cite{Papyan2019,Sagun2016,Laurent2018,Nakkiran2020}.

Despite extensive spectral evidence, there remains no single, intrinsic,
reparametrization-invariant scalar quantity that measures how empirical
sensitivity deviates from the Fisher baseline. Existing notions of effective
dimensionality (e.g., participation ratio, effective rank, spectral entropy) are
useful but do not compare empirical covariance to the intrinsic geometric
expectation imposed by Fisher--Rao structure.

\textbf{This work introduces a coordinate-free scalar diagnostic}:
\[
\mathcal{A}(\theta;q) = \mathrm{Tr}\!\left(G^{-1}C\right) - D,
\]
together with a rectified amplitude
\[
\phi(\theta;q) = \max\{\sqrt{\mathcal{A}},0\}.
\]

We show that this diagnostic:
\begin{itemize}
    \item measures the total empirical deviation from Fisher geometry,
    \item admits a spectral form $\mathcal{A} = \sum_i (\lambda_i - 1)$,
    \item distinguishes reinforcement from suppression through its sign,
    \item collapses high-dimensional spectral behavior into a single invariant.
\end{itemize}

The remainder of the paper is organized as follows.
Section~\ref{sec:background} reviews Fisher geometry and empirical covariance.
Section~\ref{sec:diagnostic} defines the alignment tensor and the scalar
diagnostic. Section~\ref{sec:spectral} develops the spectral representation.
Section~\ref{sec:regimes} characterizes equilibrium, suppression, and excess
alignment. Section~\ref{sec:experiments} presents empirical demonstrations.
Section~\ref{sec:discussion} discusses implications and limitations, and
Section~\ref{sec:conclusion} concludes.
