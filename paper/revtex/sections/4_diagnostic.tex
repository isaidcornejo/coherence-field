\section{The Alignment Tensor and Scalar Diagnostic}
\label{sec:diagnostic}

This section introduces the core geometric objects used throughout the paper:
the empirical-alignment tensor, the scalar deviation diagnostic
$\mathcal{A}(\theta;q)$, and its rectified amplitude $\phi(\theta;q)$.
Together, these quantify how empirical sensitivity departs from the intrinsic
Fisher--Rao structure \cite{AmariNagaoka2000,Amari2016}.

% -------------------------------------------------------------
\subsection{Empirical Score Covariance}

Empirical expectations with respect to the data distribution $q(x)$ define the
empirical score covariance:
\begin{equation}
C_{ij}(\theta;q) = \mathbb{E}_{q}[\,v_i v_j\,].
\end{equation}
In general $C_{ij}(\theta;q)$ and $G_{ij}(\theta)$ need not coincide unless
$q=p$, and even then equality may fail outside minimal exponential families
\cite{Brown1986,WainwrightJordan2008}.

The discrepancy between these two tensors captures the geometric deformation
induced by the data.

% -------------------------------------------------------------
\subsection{The Alignment Tensor}

We define the empirical-alignment tensor
\begin{equation}
\Delta_{ij}(\theta;q) = C_{ij}(\theta;q) - G_{ij}(\theta).
\end{equation}

Interpretationally:
\begin{itemize}
    \item $\Delta_{ij} = 0$ indicates Fisher-equivalent sensitivity.
    \item Positive eigenvalues of $\Delta$ correspond to reinforcement
    (empirical variance exceeding the Fisher baseline).
    \item Negative eigenvalues correspond to suppression.
\end{itemize}

Thus $\Delta_{ij}$ encodes direction-wise deviations of empirical sensitivity.

% -------------------------------------------------------------
\subsection{The Scalar Diagnostic}
\label{sec:scalar}

To obtain an invariant, coordinate-free quantity, we contract the alignment
tensor with the inverse Fisher metric:
\begin{equation}
\mathcal{A}(\theta;q)
    := G^{ij}(\theta)\,\Delta_{ij}(\theta;q)
    = \mathrm{Tr}\!\left(G^{-1} C\right) - D.
\end{equation}
\footnote{
The equivalence between the contractions 
$G^{ij}(C_{ij}-G_{ij})$ and $\mathrm{Tr}(G^{-1}C)-D$
follows from the identity $G^{ij}G_{ij} = \delta^{i}_{\,i} = D$,
which states that the trace of the Fisher metric equals the
dimension of the parameter space. This holds in any coordinate
system and does not assume a special parametrization.
}

This scalar diagnostic has three essential properties:
\begin{itemize}
    \item \textbf{Reparametrization invariance}: it transforms as a scalar under
    smooth parameter transformations.
    \item \textbf{Signed structure}: $\mathcal{A}>0$ indicates net reinforcement,
    whereas $\mathcal{A}<0$ indicates net suppression.
    \item \textbf{Nontriviality}: $\mathcal{A}=0$ does not imply $C_{ij}=G_{ij}$
    unless the model satisfies strict exponential-family regularity
    \cite{Brown1986}.
\end{itemize}

% -------------------------------------------------------------
\subsection{Rectified Excess-Alignment Amplitude}

In many settings only the \emph{excess-alignment sector} is of interest—namely
the directions along which empirical sensitivity exceeds the Fisher baseline.
We therefore introduce the rectified amplitude:
\begin{equation}
\phi(\theta;q) = \max\{\sqrt{\mathcal{A}(\theta;q)},\,0\}.
\end{equation}

By construction:
\[
\phi = 0 
\quad\Longleftrightarrow\quad
\mathcal{A} \le 0.
\]

The quantity $\phi$ behaves analogously to an amplitude that captures the total
strength of empirical reinforcement across all Fisher-orthonormal directions.
The square-root scaling has a natural interpretation: since 
$\mathcal{A} = \sum_i(\lambda_i - 1)$ aggregates deviations additively, taking
$\sqrt{\mathcal{A}}$ produces a Fisher-normalized magnitude rather than an
energy-like sum. In this sense, $\phi$ acts as an amplitude associated with the
subspace where $\lambda_i > 1$, providing a more interpretable measure of
excess alignment intensity.

% -------------------------------------------------------------
\subsection{Signed Interpretation}

The sign of $\mathcal{A}$ carries direct geometric meaning:
\begin{align*}
\mathcal{A} > 0: &\quad \text{net empirical reinforcement}, \\
\mathcal{A} = 0: &\quad \text{Fisher-equivalent sensitivity}, \\
\mathcal{A} < 0: &\quad \text{net empirical suppression}.
\end{align*}

Reinforcement corresponds to directions in which empirical curvature exceeds
Fisher curvature, while suppression identifies damped or collapsed empirical
modes.

These features become especially transparent through the spectral
representation developed in Section~\ref{sec:spectral}.

